\documentclass[conference]{IEEEtran}

\hyphenation{op-tical net-works semi-conduc-tor}


\begin{document}
\title{Your Project Title Here}

% author names and affiliations
% use a multiple column layout for up to three different
% affiliations
\author{\IEEEauthorblockN{Juan Aceron, Marc Teves, Martin Thomas}
\IEEEauthorblockA{Department of Computer Science\\College of Engineering\\University of the Philippines, Diliman}}

\maketitle

\begin{abstract}
	Assessing the severity of heart disease in a patient is important.
	Machine learning algorithms can solve this problem by considering risk factors and their correlation with the presence of heart disease.
	Most machine learning algorithms suffer the curse of dimensionality, wherein the time taken to fit a model with training data suffers as the number of features increases.
	In addition, having more features tends to overfit the model with the data.
	To avoid the problems with excessive features, feature selection, also known as feature ranking, is employed to reduce features in datasets with large amounts of features.
	The group uses the naive approach of feature selection, evaluating the performance of each element in the powerset of the feature set.
	Performance is taken as the weighted average of each non-diagonal cell in a confusion matrix, so that more severse false negative cases have a higher weight.
	In the case of assessing heart disease, each feature is extracted using medical procedures with varying cost.
	This is taken into account when selecting the best feature set, so as to obtain the feature set that is most effective computationally, and in terms of cost.
	The group found that (feature set) is the most effective in terms of performance, while (feature set) is the most cost effective.
\end{abstract}

\section{Introduction}
Discuss here the motivation for this topic. What makes it useful?
\section{Short of Review of Related Studies}
Discuss here references and techniques that are related to your topic. Cite sources properly.
\section{Methodology and Results}
Discuss the dataset, preprocessing, machine learning technique you used, training parameters set, performance measure and validation method.
Include tables, flowcharts, and figures, as necessary.

Discuss the performance of the model. Do the analysis here. Justify the setup. Show the experimental values vis-a-vis parameter values. Contrast with results of previous studies, if any. 

Discuss some specific cases of pitfalls (misclassification, errors) and possible reasons behind them.
\section{Conclusion}
The conclusion goes here. What were you able to accomplish? Were there any significant improvements from the previous studies reviewed? Were you able to build a model to address the topic?

\begin{thebibliography}{1}

\bibitem{IEEEhowto:kopka}
H.~Kopka and P.~W. Daly, \emph{A Guide to \LaTeX}, 3rd~ed.\hskip 1em plus
  0.5em minus 0.4em\relax Harlow, England: Addison-Wesley, 1999.

\end{thebibliography}




% that's all folks
\end{document}


